\documentclass[10pt]{article}

\usepackage[a4paper,
            bindingoffset=0.2in,
            left=1in,
            right=1in,
            top=1in,
            bottom=1in,
            footskip=.25in]{geometry}
\usepackage{mathtools}
\usepackage{amsmath}
\usepackage{amssymb}
\usepackage{enumitem}
\usepackage{environ}
\usepackage{tikz}
\usepackage{graphicx}
\graphicspath{ {./images/} }

\NewEnviron{suneq}{%
\begin{equation*}\begin{split}
  \BODY
\end{split}\end{equation*}
}

\DeclareMathOperator{\Prob}{\text{P}}
\DeclareMathOperator{\E}{\text{E}}
\DeclareMathOperator{\Var}{\text{Var}}

\DeclarePairedDelimiter\abs{\lvert}{\rvert}%
\DeclarePairedDelimiter\norm{\lVert}{\rVert}%

\newcommand*{\Perm}[2]{{}^{#1}\!P_{#2}}%
\newcommand*{\Comb}[2]{{}^{#1}C_{#2}}%

\begin{document}

\title{Modelling Uncertainty 1D Project}
\author{\b{SC06 Group 4}\\Guo Ziniu, Yeoh Siew Ning, Devon Cheng, Yong Zheng Yew}


\maketitle

\section*{Q0.}

\begin{center}
    \begin{tabular}{ c c }
        Yong Zheng Yew & 1005155 \\ 
        Yeoh Siew Ning & 1005471 \\  
        Devon Cheng & 1005215 \\
        Guo Ziniu & 1004890 \\
        Yeo Kai Wen & 1005291
    \end{tabular}
\end{center}

\begin{suneq}
    p_0 &= \frac{\frac{1005155+1005471+1005215+1004890+1005291}{5}-10^6}{10^4} \\
    &= 0.52044
\end{suneq}

\section*{Q1.}

\subsection*{a)}
\begin{suneq}
    F(n,m,p) = \sum_{k=0}^{m-1}\Comb{n-1+k}{k}\cdot p^{n}\left(1-p\right)^{k}
\end{suneq}

\subsection*{b)}
\begin{suneq}
    F(8,10,p_0) = 0.743459444891
\end{suneq}

\section*{Q2.}
\subsection*{a)}
\subsubsection*{}
Claim 1:

\subsubsection*{}
Claim 2:

\subsection*{b)}
\begin{suneq}
    F(n,m,p) = \sum_{k=n}^{n+m-1}\Comb{n+m-1}{k}\cdot p^{k}\left(1-p\right)^{\left(n+m-1-k\right)}
\end{suneq}


\section*{Q3.}
\subsection*{a)}
Let $G$ represent the event of player $A$ eventually winning the entire game, where the game in question has so far only progressed until a particular round such that $A$ and $B$ need $n$ resp. $m$ more rounds to win (or more concisely, such that $P(G)=F(n,m,p)$). Let $R$ represent the event of a player winning that particular round. According to the rules of the game, at each round, either A or B must win. Therefore, $R$ partitions the sample space between two disjoint events:
\begin{suneq}
    R = \{R_A, R_B\}
\end{suneq}
Where $R_A$ and $R_B$ represent the events of player $A$ resp. $B$ winning that round. This is because there is no way for a round to be won by both players at once (disjoint), nor can any round have any outcome other than either $A$ or $B$ winning it (partitioning).\\\\According to the law of total probability:
\begin{suneq}
    P(G) &= \sum_n P(G\cap R_n)
\end{suneq}
By Bayes' Law,
\begin{suneq}
    P(G) &= \sum_n P(G|R_n)P(R_n)\\
    &= P(G|R_A)P(R_A) + P(G|R_B)P(R_B)
\end{suneq}
We know that $P(R_A)=p$ and $P(R_B)=(1-p)$ by definition. Also, $P(G|R_A)$ and $P(G|R_B)$ respectively are the events that player $A$ eventually wins the whole game, given that player $A$ resp. $B$ wins that particular round. As shown in Question 1b), we can determine $P(G|R_A) = F(n-1,m,p)$ and $P(G|R_B) = F(n,m-1,p)$. This is because the probability of player $A$ winning a particular game can be calculated entirely from the current gamestate, and does not depend on history of past gamestates. Therefore, continuing from above, we find that:
\begin{suneq}
    P(G) &= F(n,m,p)\\
    &= p\cdot F(n-1,m,p) + (1-p)\cdot F(n,m-1,p)\\
\end{suneq}
Finally, we note that when $A$ has no more points left to win while $B$ still does, then $A$ has just won by definition; so $P(G)=F(0,m,p)=1$. Similarly, when $B$ has no more points left to win while $A$ still does, $B$ has won, meaning $A$ has lost; thus $P(G)=F(n,0,p)=0$. Thus, we have established our base cases for a recursive solution to $F(n,m,p)$:
\begin{suneq}
    F(0,m,p) &= 1 \\
    F(n,0,p) &= 0
\end{suneq}

\subsection*{b)}
Refer to Excel.

\section*{Q4.}
Let $X_{player}$ be a random variable representing the number of turns that the game ends at (inclusive), with a win for that player.
\begin{suneq}
    P(X_A = x) &= \Comb{x-1}{x-4}\cdot p^{4}\left(1-p\right)^{\left(x-4\right)}\\
    P(X_B = x) &= \Comb{x-1}{x-4}\cdot p^{\left(x-4\right)}\left(1-p\right)^{4}\\
    X &= X_A + X_B
\end{suneq}
Let $X$ be a random variable representing the number of turns that the game ends at (inclusive), with a win for player A OR B. Since $X_A$ and $X_B$ are disjoint (Players A and B cannot both win at the same time),\\
\begin{suneq}
    P(X = x) &= P((X_A = x) \cup (X_B = x))\\
    &= P(X_A = x) + P(X_B = x)\\
    &= \Comb{x-1}{x-4}\cdot\left(p^{4}\left(1-p\right)^{\left(x-4\right)}+p^{\left(x-4\right)}\left(1-p\right)^{4}\right)
\end{suneq}

\begin{center}
    \begin{tabular}{ c || c | c | c | c }
        $x$ & 4 & 5 & 6 & 7\\
        \hline
        $P(X=x)$ & 0.126253729903 & 0.250833492582 & 0.311976886701 & 0.310935890814
    \end{tabular}
\end{center}
From the above, we can calculate that
\begin{suneq}
    E(X) &= 4\left(0.126253729903\right)\ +\ 5\left(0.250833492582\right)\ +\ 6\left(0.311976886701\right)\ +\ 7\left(0.310935890814\right)\\
    &= 5.81\text{ (to 3 s.f.)}
\end{suneq}

\section*{Q5.}
\begin{suneq}
    X&\sim\text{Binomial}(2n-1,p)\\
    \therefore E(X) = (2n-1)p,\quad\Var(X) &= (2n-1)p(1-p),\quad\sigma(X) = \sqrt{\Var(X)} = \sqrt{(2n-1)p(1-p)}\\
    \text{Distance }D\text{ between }n\text{ and }E(X) &=\frac{n-E(X)}{\sigma(X)} = \abs{\frac{n-(2n-1)p}{\sqrt{(2n-1)p(1-p)}}}\\
    &=\abs{\frac{n(1-2p)+p}{\sqrt{(2n-1)p(1-p)}}}\\
    &= \abs{\frac{n(1-2p)}{\sqrt{(2n-1)p(1-p)}} + \frac{p}{\sqrt{(2n-1)p(1-p)}}}\\
    \text{As }n\text{ tends to infinity,} & \frac{p}{\sqrt{(2n-1)p(1-p)}}\to 0\\
    \therefore D&\to\abs{\frac{n(1-2p)}{\sqrt{(2n-1)p(1-p)}}} = \abs{\frac{n(2p-1)}{\sqrt{(2n-1)p(1-p)}}}\\
    &= \abs{n\cdot\frac{(2p-1)}{\sqrt{(2n-1)p(1-p)}}}
\end{suneq}

Since $\frac{(2p-1)}{\sqrt{(2n-1)p(1-p)}}$ is always positive for $0.5<p<1$, we see that, as $n$ increases, $D$ increases.

\subsection*{b)}
From a picture, we can see that $E(X)$ is to the right of $n$. Since the distance $D$ between the two is increasing, $E(X)$ is getting further away from $n$. In plain English, this means that the amount of rounds player A is expected to win is increasing faster than the minimum amount of rounds player A has to win n order to win the game. Therefore, as $n$ increases, the probability that player A will win the game increases.

\section*{Q6.}
\begin{center}
    Game State Space Diagram:\\
    \includegraphics[width=10cm]{GameSpaceDiagram.jpg}\\
\end{center}
Probability of player A winning 4 points and player B winning $k\le2$ points (so player A wins immediately):
\begin{suneq}
    \sum_{k=0}^{2}\Comb{3+k}{3}\cdot p^{4}\left(1-p\right)^{k}
\end{suneq}
Probability of both players A and B winning 3 points each:
\begin{suneq}
    \Comb{3+3}{3}\cdot p_0^3(1-p_0)^3
\end{suneq}
If both players are at 3 points each, what matters from then on is only the margin between the two. For each round, the probabilities of:
\begin{itemize}
    \item A wins: $p_0^2$
    \item B wins: $(1-p_0)^2$
    \item Inconclusive: $\Comb{2}{1}\cdot p_0^1(1-p_0)^1 = 2p_0(1-p_0)$
\end{itemize}
Therefore, the probability of player A winning the game under these circumstances is:
\begin{suneq}
    p_0^2 + 2p_0(1-p_0)p_0^2 + [2p_0(1-p_0)]^2p_0^2 + [2p_0(1-p_0)]^3p_0^2\cdots &= p_0^2 \sum_{i=0}^{\infty}(2p_0(1-p_0))^i\\
    &= \frac{p_0^2}{1-2p_0+2p_0^2}
\end{suneq}
Therefore, the total probability that player A wins the game is:
\begin{suneq}
    \sum_{k=0}^{2}\Comb{3+k}{3}\cdot p^{4}\left(1-p\right)^{k} + \Comb{3+3}{3}\cdot p_0^3(1-p_0)^3\cdot \frac{p_0^2}{1-2p_0+2p_0^2}
\end{suneq}
\end{document}